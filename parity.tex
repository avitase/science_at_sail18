\section{Spiegelsymmetrie}

\begin{frame}{Spiegelsymmetrie}
    \centering
    \textbf{Eines der beiden Bilder wurde gespiegelt ...}
    \vspace*{.5cm}

	\begin{columns}[T]
		\begin{column}{.5\textwidth}
			\centering
			\begin{overpic}[width=\textwidth]{img/rostock.png}
			\end{overpic}\\
            \only<1>{\phantom{(original)}}%
            \only<2>{(original)}
		\end{column}
		\begin{column}{.5\textwidth}
			\centering
            \reflectbox{%
			\begin{overpic}[width=\textwidth]{img/rostock.png}
            \end{overpic}}\\
            \only<1>{\phantom{(gespiegelt)}}%
            \only<2>{(gespiegelt)}
		\end{column}
	\end{columns}
    \hfill \textbf{... welches?}
\end{frame}

\begin{frame}{Spiegelsymmetrie}
    \centering
    \textbf{Eines der beiden Bilder wurde gespiegelt ...}
    \vspace*{.5cm}

	\begin{columns}[T]
		\begin{column}{.5\textwidth}
			\centering
            \reflectbox{%
			\begin{overpic}[width=\textwidth,trim=10 0 10 0,clip]{img/ambulance.png}
            \end{overpic}}\\
            \only<1>{\phantom{(gespiegelt)}}%
            \only<2>{(gespiegelt)}
		\end{column}
		\begin{column}{.5\textwidth}
			\centering
			\begin{overpic}[width=\textwidth,trim=10 0 10 0,clip]{img/ambulance.png}
			\end{overpic}\\
            \only<1>{\phantom{(original)}}%
            \only<2>{(original)}
		\end{column}
	\end{columns}
    \scalebox{.4}{(\textbf{Foto:} Ingy The Wingy)}\\
    \hfill \textbf{... welches?}
\end{frame}

\begin{frame}{Spiegelsymmetrie}
    \centering
    \textbf{Eines der beiden Bilder wurde gespiegelt ...}
    \vspace*{.5cm}

	\begin{columns}[T]
		\begin{column}{.5\textwidth}
			\centering
			\begin{overpic}[width=\textwidth]{img/mountains.png}
			\end{overpic}\\
            \only<1>{\phantom{(original)}}%
            \only<2>{(original)}
		\end{column}
		\begin{column}{.5\textwidth}
			\centering
            \reflectbox{%
			\begin{overpic}[width=\textwidth]{img/mountains.png}
            \end{overpic}}\\
            \only<1>{\phantom{(gespiegelt)}}%
            \only<2>{(gespiegelt)}
		\end{column}
	\end{columns}
    \hfill \textbf{... welches?}
\end{frame}

\begin{frame}{Spiegelsymmetrie}
    \centering
    \textbf{Eines der beiden Bilder wurde gespiegelt ...}
    \vspace*{.5cm}

	\begin{columns}[T]
		\begin{column}{.5\textwidth}
			\centering
			\begin{overpic}[width=\textwidth]{img/donkey.png}
			\end{overpic}\\
            \only<1>{\phantom{(original)}}%
            \only<2>{(original)}
		\end{column}
		\begin{column}{.5\textwidth}
			\centering
            \reflectbox{%
			\begin{overpic}[width=\textwidth]{img/donkey.png}
            \end{overpic}}\\
            \only<1>{\phantom{(gespiegelt)}}%
            \only<2>{(gespiegelt)}
		\end{column}
	\end{columns}
    \hfill \textbf{... welches?}
\end{frame}

%\begin{frame}{Von Spiegelsymmetrien zur Parit\"atsverletzung}
%    \begin{columns}[T]
%		\begin{column}{.6\textwidth}
%        Test auf \textbf{Spiegelsymmetrie}
%        \begin{itemize}
%            \item halbiere Objekt entlang einer Ebene
%            \item halte Spiegel an Schnittebene
%            \item gleichen sich das kombinierte Bild aus geteiltem Objekt \& Spiegelbild und unzerschnittenes Objekt?
%            \begin{itemize}
%                \item ja: {\color{vertexDarkRed}Spiegelsymmetrie}!
%                \item nein: keine Spiegelsymmetrie (entlang dieser Schnittebene)
%            \end{itemize}
%        \end{itemize}
%		\end{column}
%		\begin{column}{.4\textwidth}
%			\centering
%			\begin{overpic}[width=\textwidth]{img/butterflies.png}
%            \end{overpic}\\
%            \scalebox{.4}{(\textbf{Foto:} Meyer A, PLoS Biology, Vol. 4/10/2006, e341)}
%		\end{column}
%	\end{columns}
%\end{frame}

\begin{frame}{Von Spiegelsymmetrien zur Parit\"atsverletzung}
    \only<1>{%
    \centering
    \scalebox{2}{\textbf{Waren die Fotos spiegelsymmetrisch?}}}
    \only<2->{%
	\begin{columns}[T]
		\begin{column}{.6\textwidth}
            \textbf{Waren die Fotos spiegelsymmetrisch?}
            \begin{itemize}
                \item {\color{vertexDarkRed}Nein!}
                \item Trotzdem waren wir bei einigen nicht in der Lage das Original und das Gespiegelte zu unterscheiden!?
                \item Checkliste:
                \begin{itemize}
                    \item Sieht das Bild \textit{logisch} aus?
                    \item Gibt es Referenzpunkte?
                \end{itemize}
            \end{itemize}
		\end{column}
		\begin{column}{.4\textwidth}
			\centering
			\begin{overpic}[height=.7\textheight,trim=200 0 200 0,clip]{img/donkey.png}
            \end{overpic}
		\end{column}
    \end{columns}}
\end{frame}

\begin{frame}{Von Spiegelsymmetrien zur Parit\"atsverletzung}
	\begin{columns}[T]
		\begin{column}{.6\textwidth}
            \textbf{Beispiel}
            \begin{itemize}
                \item Wir \textbf{wissen}, dass der gr\"une Leuchtturm \textbf{links} von der Hafenausfahrt steht.
                \item Hier steht er aber \textbf{rechts}!
                \item Also wurde das Bild gespiegelt!
            \end{itemize}

            \only<2,3>{%
            \vspace*{.5cm}
            \textbf{... und der Esel?}
            \only<3>{%
            \begin{itemize}
                \item Hier fehlt ein Referenzpunkt!
            \end{itemize}}}
		\end{column}
		\begin{column}{.4\textwidth}
			\centering
            \only<1,2>{%
            \reflectbox{%
			\begin{overpic}[height=.7\textheight,trim=200 0 200 0,clip]{img/rostock.png}
            \end{overpic}}}%
            \only<3>{%
			\begin{overpic}[height=.7\textheight,trim=200 0 200 0,clip]{img/donkey.png}
            \end{overpic}}
		\end{column}
    \end{columns}
\end{frame}

\begin{frame}{Von Spiegelsymmetrien zur Parit\"atsverletzung}
    \textbf{Offenbar gibt es einen Unterschied zwischen:}
    \begin{itemize}
        \item etw. sieht im Spiegel exakt \textbf{gleich} aus: \textcolor{vertexDarkRed}{\textit{Spiegelsymmetrisch}}
        \item etw. sieht im Spiegel \textbf{logisch} aus: \textcolor{vertexDarkRed}{\textit{Parit\"atserhaltend}}
    \end{itemize}
    \vspace*{.5cm}
    \centering
    \textbf{Gibt es Bilder / Prozesse, die gespiegelt {\color{vertexDarkRed}\textit{unlogisch}} aussehen?}
\end{frame}

\begin{frame}{Von Spiegelsymmetrien zur Parit\"atsverletzung}
    \begin{center}
        \textbf{Gibt es Bilder / Prozesse, die gespiegelt {\color{vertexDarkRed}\textit{unlogisch}} aussehen?}
    \end{center}

	\begin{columns}[T]
		\begin{column}{.6\textwidth}
            \textbf{Sieht das Bild unlogisch aus?}
            \begin{itemize}
                \item<1-> w\"are es eine beliebige Hafeneinfahrt: {\color{vertexDarkRed}\textit{nein!}}
                \item<1-> ... wir \textbf{wissen} blo\ss{}, dass der gr\"une Leuchtturm in Rostock \textbf{links} von der Hafenausfahrt steht.
                \item<2-> K\"onnte man denn (theoretisch) die Hafenausfahrt genauso bauen, wie wir sie im gespiegelten Bild sehen? {\color{vertexDarkRed}\textit{ja!}}
                \item<3-> Die Hafenausfahrt von Rostock ist damit \textbf{parit\"atserhaltend}
            \end{itemize}
		\end{column}
		\begin{column}{.4\textwidth}
			\centering
            \reflectbox{%
			\begin{overpic}[height=.7\textheight,trim=200 0 200 0,clip]{img/rostock.png}
            \end{overpic}}
		\end{column}
    \end{columns}
\end{frame}

\begin{frame}{Suche nach Parit\"atsverletzung}
    \centering
    \scalebox{2}{%
        \textbf{Gibt es parit\"atsverletzende Prozesse?}
    }
    \begin{textblock*}{5cm}(10cm,6cm)
        \only<2->{%
        \begin{overpic}[height=2cm]{img/dracula.png}
        \end{overpic}\\
        \scalebox{.4}{(\textbf{Bild:} \texttt{https://www.lego.com} / \textit{low-resolution})}}
    \end{textblock*}
\end{frame}

\begin{frame}{Suche nach Parit\"atsverletzung}
	\begin{columns}[T]
		\begin{column}{.6\textwidth}
            \textbf{Gedankenspiel}: Kommunikation mit einem Alien
            \begin{itemize}
                \item<1-> Wir wollen herausfinden ob sein Universum \textit{gespiegelt} ist ...
                \item<1-> ... \textbf{ohne} vorbei fliegen zu m\"ussen.
                \item<2-> K\"onnen wir ihm ein Experiment vorschlagen mit dem wir unsere beiden Definitionen von \textbf{links} und \textbf{rechts} vergleichen k\"onnen?
                \begin{itemize}
                    \item<2-> ja: {\color{vertexDarkRed}Parit\"atsverletzung}!
                    \item<2-> nein: Natur / Universum ist parit\"atserhaltend
                \end{itemize}
            \end{itemize}
		\end{column}
		\begin{column}{.4\textwidth}
			\centering
            \begin{overpic}[height=.7\textheight]{img/et.png}
            \end{overpic}\\
            \scalebox{.4}{(\textbf{Poster:} John Alvin. (c) 1982 Universal Studios / \textit{low-resolution})}
		\end{column}
    \end{columns}
\end{frame}

\begin{frame}{Suche nach Parit\"atsverletzung}
    \begin{center}
        \textbf{Ein Beispiel aus der Literatur}
    \end{center}

	\begin{columns}[T]
		\begin{column}{.6\textwidth}
            \enquote{Movements of the Lower Jaw of Cattle during Mastication.}\\
            {\footnotesize P. Jordan and R. de L. Kronig. \textit{Nature} \textbf{120}, 809 (1927):}
            \begin{itemize}
                \item \enquote{[...] we shall denote as right- and left-circular cows those of which the chewing motion, viewed from the front, turns clockwise and counterclockwise respectively.}
                \item \enquote{Statistical investigations on cows distributed over the northern part of S\ae{}lland, Denmark, led to the result that about \textbf{fifty-five per cent. were right-circular}, the rest left-circular animals.}
            \end{itemize}
		\end{column}
		\begin{column}{.4\textwidth}
			\centering
			\begin{overpic}[height=.7\textheight]{img/cattlemastication_nature1927.png}
            \end{overpic}
		\end{column}
    \end{columns}
\end{frame}

\begin{frame}{Suche nach Parit\"atsverletzung}
	\begin{columns}[T]
		\begin{column}{.6\textwidth}
            \textbf{Sind K\"uhe also parit\"atsverletzend?}
            \begin{itemize}
                \item<1-> Kauen gespiegelte K\"uhe dominant in die entgegengesetzte Richtung?
                \begin{itemize}
                    \item<1-> Ja: K\"uhe sind \textbf{nicht parit\"atsverletzend}
                    \item<1-> Nein: \textbf{definiere links} als dominante Kaurichtung von K\"uhen und bitte Alien das selbe Experiment durchzuf\"uhren$\color{vertexDarkRed}{}^\star$!
                \end{itemize}
                \item<2-> Technisch schwierig umsetzbar: Kuh muss vollst\"andig (mind. molekular) gespiegelt werden ...
            \end{itemize}
		\end{column}
		\begin{column}{.4\textwidth}
			\centering
			\begin{overpic}[height=.7\textheight]{img/cow.png}
            \put(50,50){\scalebox{4}{\textcolor{white}?}}
            \put(30,70){\scalebox{1.5}{\textcolor{white}?}}
            \put(50,70){\scalebox{1.5}{\textcolor{white}?}}
            \put(65,75){\scalebox{1.}{\textcolor{white}?}}
            \end{overpic}
		\end{column}
    \end{columns}
    {\tiny $\color{vertexDarkRed}{}^\star$das Alien muss nat\"urlich seine eigenen K\"uhe verwenden!}
\end{frame}

\begin{frame}{Suche nach Parit\"atsverletzung}
    \textbf{Ein praktikablerer Ansatz:}
    \begin{itemize}
        \item Betrachte experimentellen Aufbau im Spiegel
        \item Baue jenes Spiegelbild exakt nach
        \item Damit sieht man jetzt \textbf{3} experimentelle Aufbauten:
        \begin{enumerate}
            \item Original
            \item Spiegelbild
            \item Nachbau des Spiegelbildes
        \end{enumerate}
        \item Test: Verhalten sich (2.) und (3.) gleich?
    \end{itemize}
\end{frame}

\begin{frame}{Suche nach Parit\"atsverletzung}
	\begin{columns}[T]
		\begin{column}{.65\textwidth}
            \textbf{Beispiel 1: Apfel f\"allt vom Baum}
            \scalebox{.75}{Original (1), Spiegelbild (2), Nachbau (3)}
            \begin{enumerate}
                \item Apfel f\"allt vom Baum
                \item Apfel f\"allt vom Baum
                \item Apfel f\"allt (immer noch) vom Baum
            \end{enumerate}
            \begin{itemize}
                \item \textbf{keine Parit\"atsverletzung!$\color{vertexDarkRed}{}^\star$}
            \end{itemize}
		\end{column}
		\begin{column}{.35\textwidth}
			\centering
			\begin{overpic}[width=\textwidth]{img/newton.png}
            \end{overpic}
            \scalebox{.4}{(\textbf{Bild:} LadyofHats)}
		\end{column}
    \end{columns}
    \tiny $\color{vertexDarkRed}{}^\star$tats\"achlich l\"asst sich allgemein zeigen, dass Gravitation \underline{nie} parit\"atsverletzend wirkt
\end{frame}

\begin{frame}{Suche nach Parit\"atsverletzung}
	\begin{columns}[T]
		\begin{column}{.6\textwidth}
            \textbf{Beispiel 2: Kompass in magnetischer Spule}
            \scalebox{.75}{Original (1), Spiegelbild (2), Nachbau (3)}
            \only<1>{%
            \begin{enumerate}
                \item Nadel zeigt nach oben
                \item Nadel zeigt nach \textbf{oben} (aber Windungsrichtung der Spule ist jetzt umgekehrt!)
                \item Nadel zeigt nach \textbf{unten} (Windungsrichtung entspricht Stromrichtung)
            \end{enumerate}
            \begin{itemize}
                \item \textbf{Parit\"atsverletzung?}
            \end{itemize}}
            \only<2>{%
            \begin{itemize}
                \item \textbf{Parit\"atsverletzung?}
                \item ist die Nadel wirklich sensitiv auf die Magnetfeldrichtung?
                \begin{itemize}
                    \item \textbf{Jein}: eine Magnetfeldnadel hat zwei entgegengesetzte Pole. Welcher davon \textit{Nord-} und \textit{S\"udpol} genant wird \textbf{ist willk\"urlich}!
                    \item Nach einmaliger (willk\"urlichen) Definition, k\"onnen alle anderen Nadeln an ersterer \textbf{geeicht} werden.
                    \item Im Spiegel sind \textit{Nord-} und \textit{S\"udpol} \textbf{vertauscht}!
                \end{itemize}
            \end{itemize}}
            \only<3>{%
            \begin{itemize}
                \item \textbf{Zusammenfassung}
                \begin{itemize}
                    \item \textbf{{\color{vertexDarkRed} keine Parit\"atsverletzung}} messbar$\color{vertexDarkRed}{}^\star$
                    \item Die Richtung der Magnetfeldlinien sind tats\"achlich parit\"atsverletzend, wir k\"onnen aber ihre absolute Richtung nicht messen!
                    \item Es reicht nicht im Spiegel \textit{etwas anderes} zu sehen als im Labor, sofern es konsistent ist!
                \end{itemize}
            \end{itemize}}
		\end{column}
		\begin{column}{.4\textwidth}
			\centering
			\begin{overpic}[width=\textwidth,trim=150 0 150 0,clip]{img/rotated_coil.png}
            \end{overpic}
            \scalebox{.4}{(\textbf{Foto:} Z\'atonyi S\'andor)}
		\end{column}
    \end{columns}
    \only<3>{\tiny $\color{vertexDarkRed}{}^\star$zumindest nicht mit el.-magn. Effekten (Magnetischen Monopole ausgeschlossen)}
\end{frame}
