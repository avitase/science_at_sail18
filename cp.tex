\section{Antimaterie}

\begin{frame}{Dirac-Gleichung}
    \begin{columns}[T]
        \begin{column}{.6\textwidth}
            \only<1>{%
            \textbf{Dirac-Gleichung}
            \begin{itemize}
                \item 1928 entdeckt Paul Dirac die \textit{Dirac-Gleichung}:
                \begin{center}
                    $(\mathrm{i} \slashed{\partial} - m) \psi = 0$ 
                \end{center}
                \item ... mit \textbf{2} (unabh\"angigen) L\"osungen
                \item Diese L\"osungen lassen sich \textit{physikalisch} als \textbf{Teilchen} und \textbf{Anti-Teilchen} interpretieren mit:
%                \item Die L\"osungen haben Eigenschaften welche sich physikalisch interpretieren lassen als:
%                \begin{itemize}
%                    \item Masse
%                    \item Elektrische Ladung
%                \end{itemize}
%                \item Diese 2 L\"osungen sind 2 \textit{Teilchen} mit
                \begin{itemize}
                    \item gleicher Masse
                    %\item gleicher, paarweise gespiegelter Ladung
                    \item paarweise gespiegelter Ladung
                \end{itemize}
                \item Beispiel (Elektron / Positron):
                \begin{itemize}
                    \item Elektron mit Ladung: $\mathbf{-1}$ (\textit{Teilchen})
                    \item Positron mit Ladung: $\mathbf{+1}$ (\textit{Anti-Teilchen})
                \end{itemize}
            \end{itemize}}
%                \begin{itemize}
%                    \item gleicher Masse
%                    \item gleicher Ladung
%                \end{itemize}
%            \end{itemize}}
%            \only<2,3>{%
%            \textbf{L\"osungen der Dirac-Gleichung}
%            \begin{itemize}
%                \item<2-> Beide L\"osungen haben die gleiche Masse, \textcolor{vertexDarkRed}{aber}
%                \begin{itemize}
%                    \item<2-> eine L\"osung hat \textbf{positive} Energie,
%                    \item<2-> eine L\"osung hat \textcolor{vertexDarkRed}{\textbf{negative}} Energie
%                \end{itemize}
%                \item<3> ... negative Energie?
%                \item<3> Entdeckung von R.~Feynman und E.~St\"uckelberg:
%                \begin{itemize}
%                    \item<3> Teilchen welche sich \textbf{r\"uckw\"arts} in der Zeit bewegen h\"atten negative Energie
%                    \item<3> Energie wird positiv, wenn wir stattdessen die Bewegungsrichtung und die Ladung spiegeln
%                    \item<3> diese Teilchen sind die \textbf{Anti-Teilchen}
%                \end{itemize}
%            \end{itemize}}
%            \only<4>{%
%            \textbf{L\"osungen der Dirac-Gleichung}
%            \begin{itemize}
%                \item Diese 2 L\"osungen sind 2 \textit{Teilchen} mit
%                \begin{itemize}
%                    \item gleicher Masse
%                    \item \sout{gleicher} paarweise gespiegelter Ladung
%                \end{itemize}
%                \item Beispiel (Elektron / Positron):
%                \begin{itemize}
%                    \item Elektron mit Ladung: $\mathbf{+1}$ (\textit{Teilchen})
%                    \item Positron mit Ladung: $\mathbf{-1}$ (\textit{Anti-Teilchen})
%                \end{itemize}
%            \end{itemize}}
%            \only<5>{%
            \only<2>{%
            \textbf{Zus\"atzliche Eigenschaft der L\"osungen}
            \begin{itemize}
                \item Ein \textit{Elektron} ist das Gleiche wie ein \textit{Positron}, wenn es sich 
                \begin{itemize}
                    \item in \textbf{gespiegelter} Richtung
                    \item mit \textbf{gespiegelter} Ladung bewegt
                \end{itemize}
                \item Mathematisch bewerkstelligt man die Umwandlung
                \begin{center}
                    Teilchen $\leftrightarrow$ Anti-Teilchen
                \end{center}
                durch Anwendung des sog. $C\!P$-Operators (Engl.: \textit{\underline{c}harge-\underline{p}arity})
            \end{itemize}}
        \end{column}
        \begin{column}{.4\textwidth}
            \centering
            \begin{overpic}[height=.7\textheight]{img/dirac.png}
            \end{overpic}
            \scalebox{.4}{(\textbf{Foto:} \texttt{http://nobelprize.org})}
        \end{column}
    \end{columns}    
\end{frame}

\begin{frame}{Antimaterie}
    \begin{columns}[T]
        \begin{column}{.6\textwidth}
            \textbf{Materie und Antimaterie}
            \begin{itemize}
                \item Jedes elementare Materieteilchen gehorcht seiner eigenen Dirac-Gleichung
                \begin{itemize}
                    \item Zu jedem Materieteilchen geh\"ort ein Materieantiteilchen (mit gleicher Masse und gespiegelter Ladung)
                \end{itemize}
                \item Ein Teilchen wird zum Anti-Teilchen durch Raum- und Ladungsspiegelung
            \end{itemize}
        \end{column}
        \begin{column}{.4\textwidth}
            \centering
            \begin{overpic}[height=.3\textheight,trim=400 400 400 100,clip]{img/donkey.png}
            \end{overpic}\\
            (Esel)\\
            \vspace*{.5cm}
            \reflectbox{%
            \begin{overpic}[height=.3\textheight,trim=400 400 400 100,clip]{img/anti_donkey.png}
            \end{overpic}}\\
            ($C\!P$-Esel)
        \end{column}
    \end{columns}    
\end{frame}

\begin{frame}{Einschub: Zusammensetzung unseres Universums}
    \begin{columns}[T]
        \begin{column}{.6\textwidth}
            \textbf{Zusammensetzung unseres Universums}
            \begin{itemize}
                \item 5\,\% Materie 
                \item 25\,\% Dunkle Materie (grav. Hinweise)
                \item 70\,\% Dunkle Energie (Ausdehnung v. Universum)
            \end{itemize}
            \hfill
            ... Antimaterie!?
        \end{column}
        \begin{column}{.4\textwidth}
            \centering
            \begin{overpic}[height=.7\textheight]{img/iceberg.png}
            \end{overpic} \\
            \scalebox{.4}{(\textbf{Bild:} Uwe Kils und Wiska Bodo)}
        \end{column}
    \end{columns}    
\end{frame}

\begin{frame}{Einschub: Zusammensetzung unseres Universums}
    \begin{columns}[T]
        \begin{column}{.6\textwidth}
            \textbf{Antimaterie}
            \begin{itemize}
                \item Antimaterie ist weder Dunkle Materie noch Dunkle Energie. Sie bildet eine eigene Kategorie und ist \textbf{sehr selten} in unserem Universum (sp\"ater dazu mehr).
                \item Produktion:
                \begin{itemize}
                    \item H\"ohenstrahlung
                    \item Teilchenbeschleuniger
                \end{itemize}
            \end{itemize}
        \end{column}
        \begin{column}{.4\textwidth}
           \centering
            \begin{overpic}[width=\textwidth,trim=300 0 400 0,clip]{img/antimatter_factory.png}
            \end{overpic}
        \end{column}
    \end{columns}    
\end{frame}

\begin{frame}{Das Wu-Experiment mit Antimaterie}
    \only<1>{%
    \begin{center}
        \scalebox{2}{\textbf{Ist $\pmb{C\!P}$ eine \textit{perfekte} Symmetrie?}}
    \end{center}}
    \only<2>{%
    \begin{columns}[T]
        \begin{column}{.6\textwidth}
            \textbf{Das Experiment im $\pmb{C\!P}$-Spiegel}
            \begin{itemize}
                \item Magnetische Spule
                \begin{itemize}
                    \item Windungsrichtung \textbf{und} Ladung werden gespiegelt
                    \item Magnetfeld \"andert seine Richtung nicht
                \end{itemize}
                \item Ausrichtung der Atome im magn. Feld:
                \begin{itemize}
                    \item h\"angt von elektrischer Ladung (inkl. Vorzeichen) ab
                    \item Magnetfeld und Atome sind jetzt also \textbf{antiparallel}
                \end{itemize}
            \end{itemize}
        \end{column}
        \begin{column}{.4\textwidth}
           \centering
            \begin{overpic}[height=.7\textheight]{img/madamewu.png}
            \end{overpic}\\
            \centering
            \scalebox{.4}{(\textbf{Photo:} Acc. 90-105 - Science Service, Records, 1920s-1970s, Smithsonian Institution Archives)}
        \end{column}
    \end{columns}}
\end{frame}

\begin{frame}{Das Wu-Experiment mit Antimaterie}
    \begin{columns}[T]
        \begin{column}{0.33\textwidth}
            \begin{overpic}[width=\textwidth]{img/wu.png}
                \put(20,50){\scalebox{5}{$\color{white}\uparrow$}}
                \put(30,20){\scalebox{2}{$\color{white}e^-$}}
                \put(40,90){\scalebox{2}{$\color{white}\overline{\nu}_{\!e}$}}
            \end{overpic}
            \center
            (Original)
        \end{column}
        \begin{column}{0.33\textwidth}
            \begin{overpic}[width=\textwidth]{img/anti_wu_flipped.png}
                \put(37,50){\scalebox{4}{\color{black}?}}
                \put(35,20){\scalebox{2}{$\color{black}e^+$}}
                \put(20,90){\scalebox{2}{$\color{black}\nu_{\!e}$}}
            \end{overpic}
            \center
            ($C\!P$-Spiegelbild)
        \end{column}
        \begin{column}{0.33\textwidth}
            \begin{overpic}[width=\textwidth]{img/anti_wu_flipped.png}
                \put(37,50){\scalebox{5}{$\color{black}\uparrow$}}
                \put(35,20){\scalebox{2}{$\color{black}e^+$}}
                \put(20,90){\scalebox{2}{$\color{black}\nu_{\!e}$}}
            \end{overpic}
            \center
            (Nachbau)
        \end{column}
    \end{columns}
    \centering
    \scalebox{.4}{(\textbf{Foto:} Annual Report of the National Bureau of Standards for 1957, miscellaneous publication 227)}
\end{frame}

\begin{frame}{Das Wu-Experiment mit Antimaterie}
    \begin{columns}[T]
        \begin{column}{.6\textwidth}
            \textbf{Auswertung}
            \begin{itemize}
                \item Problem im (Parit\"ats-)Spiegel
                \begin{itemize}
                    \item die Atomausrichtung kehrt sich um (umgekehrte Magnetfeldrichtung)
                    \item $e^-$-Emission \textbf{entgegen} der Atomausrichtung, sowohl im Original, als auch im Nachbau
                \end{itemize}
                \item<2> ... im $C\!P$-Spiegel
                \begin{itemize}
                    \item<2> die (Anti-)Atomausrichtung kehrt sich um (gleiche Magnetfeldrichtung)
                    \item<2> $e^+$-Emission \textbf{entlang} der \textbf{Anti-}Atomausrichtung
                \end{itemize}
            \end{itemize}
        \end{column}
        \begin{column}{.4\textwidth}
            \centering
            \only<1>{%
            \begin{overpic}[height=.7\textheight]{img/wu.png}
                \put(20,50){\scalebox{5}{$\color{white}\uparrow$}}
                \put(30,20){\scalebox{2}{$\color{white}e^-$}}
                \put(40,90){\scalebox{2}{$\color{white}\overline{\nu}_{\!e}$}}
            \end{overpic}\\
            (Original)}%
            \only<2>{%
            \begin{overpic}[height=.7\textheight]{img/anti_wu_flipped.png}
                \put(37,50){\scalebox{5}{$\color{black}\uparrow$}}
                \put(35,20){\scalebox{2}{$\color{black}e^+$}}
                \put(20,90){\scalebox{2}{$\color{black}\nu_{\!e}$}}
            \end{overpic}\\
            (Nachbau)}
        \end{column}
    \end{columns}
\end{frame}

\begin{frame}{Das Wu-Experiment mit Antimaterie}
    \begin{columns}[T]
        \begin{column}{.6\textwidth}
            \textbf{Gedankenspiel}: Kommunikation mit einem Alien
            \begin{itemize}
                \item Wir wollen herausfinden ob sein Universum \textit{\sout{gespiegelt} aus Antimaterie aufgebaut} ist ...
                \item ... \textbf{ohne} vorbei fliegen zu m\"ussen.
                \item K\"onnen wir ihm ein Experiment vorschlagen mit dem wir unsere beiden Definitionen von \textbf{\sout{links} Materie} und \textbf{\sout{rechts} Antimaterie} vergleichen k\"onnen?
                \begin{itemize}
                    \item Offenbar nicht mit dem Wu-Experiment (mit Antimaterie)!
					\item Ist unsere Definition von Materie und Antimaterie als \textbf{willk\"urlich}?
                \end{itemize}
            \end{itemize}
        \end{column}
        \begin{column}{.4\textwidth}
            \centering
            \begin{overpic}[height=.7\textheight]{img/et.png}
            \end{overpic}\\
            \scalebox{.4}{(\textbf{Poster:} John Alvin. (c) 1982 Universal Studios / \textit{low-resolution})}
        \end{column}
    \end{columns}
\end{frame}
