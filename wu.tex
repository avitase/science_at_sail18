\section{Wu-Experiment}

\begin{frame}{Das Wu-Experiment}
    \begin{block}{Physik im Jahre 1956}
        \begin{itemize}
            \item Parit\"at ist erhalten in 3 der 4 bekannten Wechselwirkungen:
            \begin{itemize}
                \item \textit{Gravitation}
                \item \textit{Elektromagnetische Wechselwirkung}
                \item \textit{Starke Wechselwirkung}
            \end{itemize}
            \item ... wie sieht es mit der \textbf{\textit{Schwachen Wechselwirkung}} aus?
            \begin{itemize}
                \item Pr\"apariere Kernzerfall: Cobalt-60 $\to$ Nickel-60 ($\beta$-Zerfall)
                \item Richte alle Kerne im \textcolor{vertexDarkRed}{magnetischen} Feld gleich aus$\color{vertexDarkRed}{}^\star$
                \item Untersuche die \textcolor{vertexDarkRed}{r\"aumliche Verteilung} der $\beta$-Strahlung
            \end{itemize}
        \end{itemize}
    \end{block}
    \tiny $\color{vertexDarkRed}{}^\star$als Auszeichnungsrichtung dient hier der sog. \textit{Kern-Spin}. Magn. Feld und \textit{Kern-Spin} sind nach Ausrichtung parallel.
\end{frame}

\begin{frame}{Einschub: Beta-Zerfall}
    \begin{itemize}
        \item \textbf{Atome} bestehen aus einem \textbf{Atomkern} und Elektronen
        \item \textbf{Atomkerne} sind aufgebaut aus \textbf{Protonen} und \textbf{Neutronen}
        \item Beim radioaktiven Zerfall wandelt sich ein Neutron in ein Proton \& Beta-Strahlung um (oder umgekehrt) 
        \begin{align*}
                n &\to p + \beta^- \quad \text{(Neutron-Zerfall)}\\
                p &\to n + \beta^+ \quad \text{(Proton-Zerfall)}
        \end{align*}
        \item \textbf{Beta-Strahlung} besteht aus zwei Teilchen: einem \textbf{Elektron} und einem \textbf{Anti-Elektron-Neutrino}
        \begin{align*}
            \beta^- &\;\hat{=}\; e^- + \overline{\nu}_e \quad \text{(Elektron + Anti-Elektron-Neutrino)}\\
            \beta^+ &\;\hat{=}\; e^+ + \nu_e \quad \text{(Positron + Elektron-Neutrino)}
        \end{align*}
    \end{itemize}
\end{frame}

\begin{frame}{Einschub: Beta-Zerfall}
    \textbf{Cobalt-60 Zerfall (nach Nickel-60)}
    \begin{itemize}
        \item Cobalt-60: 27 Protonen, 33 Neutronen
        \item Nickel-60: 28 Protonen, 32 Neutronen
        \item Also offenbar: Neutron $\to$ Proton ($\beta^-$ Strahlung, $\beta^- \;\hat{=}\; e^- + \overline{\nu}_e$)
        \item Das neue Proton nimmt den Platz vom zerfallenem Neutron ein. Das Elektron und das Anti-Elektron-Neutrino fliegen als $\beta^-$ Strahlung in \textbf{entgegengesetzte} Richtungen (Impulserhaltung).
    \end{itemize}
\end{frame}

\begin{frame}{Das Wu Experiment}
    \begin{columns}[T]
        \begin{column}{0.33\textwidth}
            \begin{overpic}[width=\textwidth]{img/wu.png}
                \put(20,50){\scalebox{5}{$\color{white}\uparrow$}}
                \put(30,20){\scalebox{2}{$\color{white}e^-$}}
                \put(40,90){\scalebox{2}{$\color{white}\overline{\nu}_{\!e}$}}
            \end{overpic}
            \center
            (Original)
        \end{column}
        \begin{column}{0.33\textwidth}
            \begin{overpic}[width=\textwidth]{img/wu_flipped.png}
                \put(37,50){\scalebox{4}{\color{white}?}}
                \put(35,20){\scalebox{2}{$\color{white}e^-$}}
                \put(20,90){\scalebox{2}{$\color{white}\overline{\nu}_{\!e}$}}
            \end{overpic}
            \center
            (Spiegelbild)
        \end{column}
        \begin{column}{0.33\textwidth}
            \begin{overpic}[width=\textwidth]{img/wu_flipped.png}
                \put(37,50){\scalebox{5}{$\color{white}\downarrow$}}
                \put(35,20){\scalebox{2}{$\color{white}\overline{\nu}_{\!e}$}}
                \put(20,90){\scalebox{2}{$\color{white}e^-$}}
            \end{overpic}
            \center
            (Nachbau)
        \end{column}
    \end{columns}
    \centering
    \scalebox{.4}{(\textbf{Foto:} Annual Report of the National Bureau of Standards for 1957, miscellaneous publication 227)}
\end{frame}

\begin{frame}{Das Wu-Experiment}
    \textbf{Das ist Parit\"atsverletzung!}
    \begin{itemize}
        \item Experimenteller Aufbau im Spiegel zeigt Elektronen Detektion \textbf{unten} (demnach m\"usste das magn. Feld nach \textbf{oben} zeigen)
        \item Nachbau des Spiegelbildes detektiert aber Elektronen \textbf{oben} (demnach m\"usste das magn. Feld nach \textbf{unten} zeigen)
        \item Warum?
        \begin{itemize}
            \item Im Vergleich zum Experiment mit dem Kompass sind das Elektron und das Anti-Elektron-Neutrino der Nord- und der S\"udpol des Kompasses. Diesmal aber \textbf{eindeutig bestimmt}!
        \end{itemize}
    \end{itemize}
\end{frame}

\begin{frame}{Das Wu-Experiment - Implikationen}
        Wir haben jetzt die M\"oglichkeit eine gespiegelte Welt \textbf{eindeutig zu definieren}:
        \begin{block}{Definition}
            Man lebt in einer gespiegelten Welt, wenn die Richtung der emittierten Elektronen im Cobalt-60 Kernzerf\"allen \textbf{parallel} (nicht anti-parallel) zu der Spin-Ausrichtung der Atomkerne ist.
        \end{block}
    \begin{itemize}
        \item Damit k\"onnen wir zwischen beiden Welten \textbf{unterscheiden} und ...
        \item ... wir k\"onnen \textbf{eindeutig bestimmen} in welcher Welt wir leben.
        \item Also ist unser Universum \textcolor{vertexDarkRed}{nicht parit\"atsinvariant}!
    \end{itemize}
\end{frame}

\begin{frame}{Das Wu-Experiment}
	\begin{columns}[T]
		\begin{column}{.6\textwidth}
            \textbf{Zusammenfassung}
            \begin{itemize}
                \item Schwache Wechselwirkung verletzt (als einzige) die Parit\"ats Invarianz
                \item Erstmals beobachtet mit dem \textit{Wu-Experiment}
                \item Parit\"ats Invarianz ist gebrochen, weil Beta-Strahlung parallel zur Magnetfeldrichtung emittiert wird
                \item Nobel Preis 1957
            \end{itemize}
		\end{column}
		\begin{column}{.4\textwidth}
			\centering
			\begin{overpic}[height=.8\textheight]{img/madamewu.png}
            \end{overpic}\\
            \centering
            \scalebox{.4}{(\textbf{Photo:} Acc. 90-105 - Science Service, Records, 1920s-1970s, Smithsonian Institution Archives)}
		\end{column}
	\end{columns}
\end{frame}
