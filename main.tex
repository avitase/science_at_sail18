\documentclass[compress,aspectratio=1610]{beamer}

\usepackage{polyglossia}
\setmainlanguage{german}
\usepackage{fontspec}
\usepackage[autostyle]{csquotes}
\usepackage{multicol}
\usepackage[percent]{overpic}
\usepackage[absolute,overlay]{textpos}
\usepackage{ulem}
\usepackage{slashed}
\usepackage{nicefrac}
\usepackage{bm}
\usepackage{siunitx}

\usetheme{vertex}

\linespread{1.1}

\title{Spiegelwelten und Antimaterie am LHC}
\subtitle{Science~@~Sail 2018}
%\institute{Universit\"at Rostock / LHCb}
%\date{12.08.2018}
\author{Nis Meinert}

\setbeamertemplate{section page}{
    \begin{center}
		\Huge\color{vertexDarkRed}\insertsection
	\end{center}
}

\AtBeginSection{
	\begin{frame}[plain,noframenumbering]
		\sectionpage
	\end{frame}
}

\setbeamertemplate{subsection page}{
	\begin{center}
		\color{vertexDarkRed}
		\Huge\insertsection \\[10pt]
		\Large\insertsubsection
	\end{center}
}

\AtBeginSubsection{
	\begin{frame}
		\subsectionpage
	\end{frame}
}

\begin{document}

\begin{frame}[plain,noframenumbering]
    \titlepage
\end{frame}

\begin{frame}{Nis Meinert}
    \begin{columns}[T]
        \begin{column}{.6\textwidth}
            \begin{itemize}
                \item LHCb Mitglied seit 2013
                \item 2016: Master of Science (Uni. Rostock)
                \begin{itemize}
                    \item Massenspektroskopie von schweren Baryonen
                \end{itemize}
                \item Seit 2016: Doktorand am Institut f\"ur Physik (Uni. Rostock)
                \item Mehrere k\"urzere Aufenthalte am CERN
                \item Forschungsschwerpunkt: 
                \begin{itemize}
                    \item $C\!P$-Verletzung in Zerf\"allen von schweren Baryonen
                    \item Upgrade des LHCb Detektors
                \end{itemize}
            \end{itemize}
        \end{column}
        \begin{column}{.4\textwidth}
            \centering
            \begin{overpic}[height=.8\textheight,trim=200 0 0 0,clip]{img/me.png}
            \end{overpic}
        \end{column}
    \end{columns}
\end{frame}

\begin{frame}{Was ist der LHC / Was ist CERN?}
    \begin{columns}[T]
        \begin{column}{.5\textwidth}
            \textbf{Der Large Hadron Collider (LHC)}
            \begin{itemize}
                \item Wikipedia: \enquote{[...] the most complex experimental facility ever built and the largest single machine in the world.}
                \item Synchrotron (in einem 27\,km langem unterirdischem Ringtunnel): \textbf{beschleunigt} u.a. Protonen auf fast Lichtgeschwindigkeit und \textbf{kollidiert} diese an bestimmten Punkten (zum Beispiel beim LHCb Detektor)
            \end{itemize}
        \end{column}
        \begin{column}{.5\textwidth}
            \centering
            \begin{overpic}[width=\textwidth]{img/cern_map.png}
            \end{overpic}
        \end{column}
    \end{columns}
\end{frame}

\begin{frame}{Was wir nicht verstehen}
    \begin{columns}[T]
        \begin{column}{.6\textwidth}
            \begin{itemize}
                \item Ein Blick in unser Universum
                \begin{itemize}
                    \item (fast) ausschlie\ss{}lich Materie und keine Antimaterie
                \end{itemize}
                \item Standard Modell der Kosmologie
                \begin{itemize}
                    \item Beim Urknall wurden exakt gleiche Mengen von Materie und Antimaterie produziert
                \end{itemize}
            \end{itemize}
            \centering
            \textbf{Wo ist die Antimaterie?}
        \end{column}
        \begin{column}{.4\textwidth}
            \centering
            \begin{overpic}[height=.7\textheight]{img/ams02.png}
            \end{overpic}\\
            \centering
            \scalebox{.4}{(\textbf{Logo:} AMS-02, NASA/JSC)}
        \end{column}
    \end{columns}
\end{frame}


\begin{frame}
    \frametitle{Gliederung}
    \centering
    \scalebox{1.5}{\textbf{\textcolor{vertexDarkRed}{1.} Spiegelsymmetrie}}\\
    \vspace{5mm}
    \scalebox{1.5}{\textbf{\textcolor{vertexDarkRed}{2.} Wu-Experiment}}\\
    \vspace{5mm}
    \scalebox{1.5}{\textbf{\textcolor{vertexDarkRed}{3.} Antimaterie}}\\
    \vspace{5mm}
    \scalebox{1.5}{\textbf{\textcolor{vertexDarkRed}{4.} CP-Verletzung}}
\end{frame}

\section{Spiegelsymmetrie}

\begin{frame}{Spiegelsymmetrie}
    \centering
    \textbf{Eines der beiden Bilder wurde gespiegelt ...}
    \vspace*{.5cm}

	\begin{columns}[T]
		\begin{column}{.5\textwidth}
			\centering
			\begin{overpic}[width=\textwidth]{img/rostock.png}
			\end{overpic}\\
            \only<1>{\phantom{(original)}}%
            \only<2>{(original)}
		\end{column}
		\begin{column}{.5\textwidth}
			\centering
            \reflectbox{%
			\begin{overpic}[width=\textwidth]{img/rostock.png}
            \end{overpic}}\\
            \only<1>{\phantom{(gespiegelt)}}%
            \only<2>{(gespiegelt)}
		\end{column}
	\end{columns}
    \hfill \textbf{... welches?}
\end{frame}

\begin{frame}{Spiegelsymmetrie}
    \centering
    \textbf{Eines der beiden Bilder wurde gespiegelt ...}
    \vspace*{.5cm}

	\begin{columns}[T]
		\begin{column}{.5\textwidth}
			\centering
            \reflectbox{%
			\begin{overpic}[width=\textwidth,trim=10 0 10 0,clip]{img/ambulance.png}
            \end{overpic}}\\
            \only<1>{\phantom{(gespiegelt)}}%
            \only<2>{(gespiegelt)}
		\end{column}
		\begin{column}{.5\textwidth}
			\centering
			\begin{overpic}[width=\textwidth,trim=10 0 10 0,clip]{img/ambulance.png}
			\end{overpic}\\
            \only<1>{\phantom{(original)}}%
            \only<2>{(original)}
		\end{column}
	\end{columns}
    \scalebox{.4}{(\textbf{Foto:} Ingy The Wingy)}\\
    \hfill \textbf{... welches?}
\end{frame}

\begin{frame}{Spiegelsymmetrie}
    \centering
    \textbf{Eines der beiden Bilder wurde gespiegelt ...}
    \vspace*{.5cm}

	\begin{columns}[T]
		\begin{column}{.5\textwidth}
			\centering
			\begin{overpic}[width=\textwidth]{img/mountains.png}
			\end{overpic}\\
            \only<1>{\phantom{(original)}}%
            \only<2>{(original)}
		\end{column}
		\begin{column}{.5\textwidth}
			\centering
            \reflectbox{%
			\begin{overpic}[width=\textwidth]{img/mountains.png}
            \end{overpic}}\\
            \only<1>{\phantom{(gespiegelt)}}%
            \only<2>{(gespiegelt)}
		\end{column}
	\end{columns}
    \hfill \textbf{... welches?}
\end{frame}

\begin{frame}{Spiegelsymmetrie}
    \centering
    \textbf{Eines der beiden Bilder wurde gespiegelt ...}
    \vspace*{.5cm}

	\begin{columns}[T]
		\begin{column}{.5\textwidth}
			\centering
			\begin{overpic}[width=\textwidth]{img/donkey.png}
			\end{overpic}\\
            \only<1>{\phantom{(original)}}%
            \only<2>{(original)}
		\end{column}
		\begin{column}{.5\textwidth}
			\centering
            \reflectbox{%
			\begin{overpic}[width=\textwidth]{img/donkey.png}
            \end{overpic}}\\
            \only<1>{\phantom{(gespiegelt)}}%
            \only<2>{(gespiegelt)}
		\end{column}
	\end{columns}
    \hfill \textbf{... welches?}
\end{frame}

%\begin{frame}{Von Spiegelsymmetrien zur Parit\"atsverletzung}
%    \begin{columns}[T]
%		\begin{column}{.6\textwidth}
%        Test auf \textbf{Spiegelsymmetrie}
%        \begin{itemize}
%            \item halbiere Objekt entlang einer Ebene
%            \item halte Spiegel an Schnittebene
%            \item gleichen sich das kombinierte Bild aus geteiltem Objekt \& Spiegelbild und unzerschnittenes Objekt?
%            \begin{itemize}
%                \item ja: {\color{vertexDarkRed}Spiegelsymmetrie}!
%                \item nein: keine Spiegelsymmetrie (entlang dieser Schnittebene)
%            \end{itemize}
%        \end{itemize}
%		\end{column}
%		\begin{column}{.4\textwidth}
%			\centering
%			\begin{overpic}[width=\textwidth]{img/butterflies.png}
%            \end{overpic}\\
%            \scalebox{.4}{(\textbf{Foto:} Meyer A, PLoS Biology, Vol. 4/10/2006, e341)}
%		\end{column}
%	\end{columns}
%\end{frame}

\begin{frame}{Von Spiegelsymmetrien zur Parit\"atsverletzung}
    \only<1>{%
    \centering
    \scalebox{2}{\textbf{Waren die Fotos spiegelsymmetrisch?}}}
    \only<2->{%
	\begin{columns}[T]
		\begin{column}{.6\textwidth}
            \textbf{Waren die Fotos spiegelsymmetrisch?}
            \begin{itemize}
                \item {\color{vertexDarkRed}Nein!}
                \item Trotzdem waren wir bei einigen nicht in der Lage das Original und das Gespiegelte zu unterscheiden!?
                \item Checkliste:
                \begin{itemize}
                    \item Sieht das Bild \textit{logisch} aus?
                    \item Gibt es Referenzpunkte?
                \end{itemize}
            \end{itemize}
		\end{column}
		\begin{column}{.4\textwidth}
			\centering
			\begin{overpic}[height=.7\textheight,trim=200 0 200 0,clip]{img/donkey.png}
            \end{overpic}
		\end{column}
    \end{columns}}
\end{frame}

\begin{frame}{Von Spiegelsymmetrien zur Parit\"atsverletzung}
	\begin{columns}[T]
		\begin{column}{.6\textwidth}
            \textbf{Beispiel}
            \begin{itemize}
                \item Wir \textbf{wissen}, dass der gr\"une Leuchtturm \textbf{links} von der Hafenausfahrt steht.
                \item Hier steht er aber \textbf{rechts}!
                \item Also wurde das Bild gespiegelt!
            \end{itemize}

            \only<2,3>{%
            \vspace*{.5cm}
            \textbf{... und der Esel?}
            \only<3>{%
            \begin{itemize}
                \item Hier fehlt ein Referenzpunkt!
            \end{itemize}}}
		\end{column}
		\begin{column}{.4\textwidth}
			\centering
            \only<1,2>{%
            \reflectbox{%
			\begin{overpic}[height=.7\textheight,trim=200 0 200 0,clip]{img/rostock.png}
            \end{overpic}}}%
            \only<3>{%
			\begin{overpic}[height=.7\textheight,trim=200 0 200 0,clip]{img/donkey.png}
            \end{overpic}}
		\end{column}
    \end{columns}
\end{frame}

\begin{frame}{Von Spiegelsymmetrien zur Parit\"atsverletzung}
    \textbf{Offenbar gibt es einen Unterschied zwischen:}
    \begin{itemize}
        \item etw. sieht im Spiegel exakt \textbf{gleich} aus: \textcolor{vertexDarkRed}{\textit{Spiegelsymmetrisch}}
        \item etw. sieht im Spiegel \textbf{logisch} aus: \textcolor{vertexDarkRed}{\textit{Parit\"atserhaltend}}
    \end{itemize}
    \vspace*{.5cm}
    \centering
    \textbf{Gibt es Bilder / Prozesse, die gespiegelt {\color{vertexDarkRed}\textit{unlogisch}} aussehen?}
\end{frame}

\begin{frame}{Von Spiegelsymmetrien zur Parit\"atsverletzung}
    \begin{center}
        \textbf{Gibt es Bilder / Prozesse, die gespiegelt {\color{vertexDarkRed}\textit{unlogisch}} aussehen?}
    \end{center}

	\begin{columns}[T]
		\begin{column}{.6\textwidth}
            \textbf{Sieht das Bild unlogisch aus?}
            \begin{itemize}
                \item<1-> w\"are es eine beliebige Hafeneinfahrt: {\color{vertexDarkRed}\textit{nein!}}
                \item<1-> ... wir \textbf{wissen} blo\ss{}, dass der gr\"une Leuchtturm in Rostock \textbf{links} von der Hafenausfahrt steht.
                \item<2-> K\"onnte man denn (theoretisch) die Hafenausfahrt genauso bauen, wie wir sie im gespiegelten Bild sehen? {\color{vertexDarkRed}\textit{ja!}}
                \item<3-> Die Hafenausfahrt von Rostock ist damit \textbf{parit\"atserhaltend}
            \end{itemize}
		\end{column}
		\begin{column}{.4\textwidth}
			\centering
            \reflectbox{%
			\begin{overpic}[height=.7\textheight,trim=200 0 200 0,clip]{img/rostock.png}
            \end{overpic}}
		\end{column}
    \end{columns}
\end{frame}

\begin{frame}{Suche nach Parit\"atsverletzung}
    \centering
    \scalebox{2}{%
        \textbf{Gibt es parit\"atsverletzende Prozesse?}
    }
    \begin{textblock*}{5cm}(10cm,6cm)
        \only<2->{%
        \begin{overpic}[height=2cm]{img/dracula.png}
        \end{overpic}\\
        \scalebox{.4}{(\textbf{Bild:} \texttt{https://www.lego.com} / \textit{low-resolution})}}
    \end{textblock*}
\end{frame}

\begin{frame}{Suche nach Parit\"atsverletzung}
	\begin{columns}[T]
		\begin{column}{.6\textwidth}
            \textbf{Gedankenspiel}: Kommunikation mit einem Alien
            \begin{itemize}
                \item<1-> Wir wollen herausfinden ob sein Universum \textit{gespiegelt} ist ...
                \item<1-> ... \textbf{ohne} vorbei fliegen zu m\"ussen.
                \item<2-> K\"onnen wir ihm ein Experiment vorschlagen mit dem wir unsere beiden Definitionen von \textbf{links} und \textbf{rechts} vergleichen k\"onnen?
                \begin{itemize}
                    \item<2-> ja: {\color{vertexDarkRed}Parit\"atsverletzung}!
                    \item<2-> nein: Natur / Universum ist parit\"atserhaltend
                \end{itemize}
            \end{itemize}
		\end{column}
		\begin{column}{.4\textwidth}
			\centering
            \begin{overpic}[height=.7\textheight]{img/et.png}
            \end{overpic}\\
            \scalebox{.4}{(\textbf{Poster:} John Alvin. (c) 1982 Universal Studios / \textit{low-resolution})}
		\end{column}
    \end{columns}
\end{frame}

\begin{frame}{Suche nach Parit\"atsverletzung}
    \begin{center}
        \textbf{Ein Beispiel aus der Literatur}
    \end{center}

	\begin{columns}[T]
		\begin{column}{.6\textwidth}
            \enquote{Movements of the Lower Jaw of Cattle during Mastication.}\\
            {\footnotesize P. Jordan and R. de L. Kronig. \textit{Nature} \textbf{120}, 809 (1927):}
            \begin{itemize}
                \item \enquote{[...] we shall denote as right- and left-circular cows those of which the chewing motion, viewed from the front, turns clockwise and counterclockwise respectively.}
                \item \enquote{Statistical investigations on cows distributed over the northern part of S\ae{}lland, Denmark, led to the result that about \textbf{fifty-five per cent. were right-circular}, the rest left-circular animals.}
            \end{itemize}
		\end{column}
		\begin{column}{.4\textwidth}
			\centering
			\begin{overpic}[height=.7\textheight]{img/cattlemastication_nature1927.png}
            \end{overpic}
		\end{column}
    \end{columns}
\end{frame}

\begin{frame}{Suche nach Parit\"atsverletzung}
	\begin{columns}[T]
		\begin{column}{.6\textwidth}
            \textbf{Sind K\"uhe also parit\"atsverletzend?}
            \begin{itemize}
                \item<1-> Kauen gespiegelte K\"uhe dominant in die entgegengesetzte Richtung?
                \begin{itemize}
                    \item<1-> Ja: K\"uhe sind \textbf{nicht parit\"atsverletzend}
                    \item<1-> Nein: \textbf{definiere links} als dominante Kaurichtung von K\"uhen und bitte Alien das selbe Experiment durchzuf\"uhren$\color{vertexDarkRed}{}^\star$!
                \end{itemize}
                \item<2-> Technisch schwierig umsetzbar: Kuh muss vollst\"andig (mind. molekular) gespiegelt werden ...
            \end{itemize}
		\end{column}
		\begin{column}{.4\textwidth}
			\centering
			\begin{overpic}[height=.7\textheight]{img/cow.png}
            \put(50,50){\scalebox{4}{\textcolor{white}?}}
            \put(30,70){\scalebox{1.5}{\textcolor{white}?}}
            \put(50,70){\scalebox{1.5}{\textcolor{white}?}}
            \put(65,75){\scalebox{1.}{\textcolor{white}?}}
            \end{overpic}
		\end{column}
    \end{columns}
    {\tiny $\color{vertexDarkRed}{}^\star$das Alien muss nat\"urlich seine eigenen K\"uhe verwenden!}
\end{frame}

\begin{frame}{Suche nach Parit\"atsverletzung}
    \textbf{Ein praktikablerer Ansatz:}
    \begin{itemize}
        \item Betrachte experimentellen Aufbau im Spiegel
        \item Baue jenes Spiegelbild exakt nach
        \item Damit sieht man jetzt \textbf{3} experimentelle Aufbauten:
        \begin{enumerate}
            \item Original
            \item Spiegelbild
            \item Nachbau des Spiegelbildes
        \end{enumerate}
        \item Test: Verhalten sich (2.) und (3.) gleich?
    \end{itemize}
\end{frame}

\begin{frame}{Suche nach Parit\"atsverletzung}
	\begin{columns}[T]
		\begin{column}{.65\textwidth}
            \textbf{Beispiel 1: Apfel f\"allt vom Baum}
            \scalebox{.75}{Original (1), Spiegelbild (2), Nachbau (3)}
            \begin{enumerate}
                \item Apfel f\"allt vom Baum
                \item Apfel f\"allt vom Baum
                \item Apfel f\"allt (immer noch) vom Baum
            \end{enumerate}
            \begin{itemize}
                \item \textbf{keine Parit\"atsverletzung!$\color{vertexDarkRed}{}^\star$}
            \end{itemize}
		\end{column}
		\begin{column}{.35\textwidth}
			\centering
			\begin{overpic}[width=\textwidth]{img/newton.png}
            \end{overpic}
            \scalebox{.4}{(\textbf{Bild:} LadyofHats)}
		\end{column}
    \end{columns}
    \tiny $\color{vertexDarkRed}{}^\star$tats\"achlich l\"asst sich allgemein zeigen, dass Gravitation \underline{nie} parit\"atsverletzend wirkt
\end{frame}

\begin{frame}{Suche nach Parit\"atsverletzung}
	\begin{columns}[T]
		\begin{column}{.6\textwidth}
            \textbf{Beispiel 2: Kompass in magnetischer Spule}
            \scalebox{.75}{Original (1), Spiegelbild (2), Nachbau (3)}
            \only<1>{%
            \begin{enumerate}
                \item Nadel zeigt nach oben
                \item Nadel zeigt nach \textbf{oben} (aber Windungsrichtung der Spule ist jetzt umgekehrt!)
                \item Nadel zeigt nach \textbf{unten} (Windungsrichtung entspricht Stromrichtung)
            \end{enumerate}
            \begin{itemize}
                \item \textbf{Parit\"atsverletzung?}
            \end{itemize}}
            \only<2>{%
            \begin{itemize}
                \item \textbf{Parit\"atsverletzung?}
                \item ist die Nadel wirklich sensitiv auf die Magnetfeldrichtung?
                \begin{itemize}
                    \item \textbf{Jein}: eine Magnetfeldnadel hat zwei entgegengesetzte Pole. Welcher davon \textit{Nord-} und \textit{S\"udpol} genant wird \textbf{ist willk\"urlich}!
                    \item Nach einmaliger (willk\"urlichen) Definition, k\"onnen alle anderen Nadeln an ersterer \textbf{geeicht} werden.
                    \item Im Spiegel sind \textit{Nord-} und \textit{S\"udpol} \textbf{vertauscht}!
                \end{itemize}
            \end{itemize}}
            \only<3>{%
            \begin{itemize}
                \item \textbf{Zusammenfassung}
                \begin{itemize}
                    \item \textbf{{\color{vertexDarkRed} keine Parit\"atsverletzung}} messbar$\color{vertexDarkRed}{}^\star$
                    \item Die Richtung der Magnetfeldlinien sind tats\"achlich parit\"atsverletzend, wir k\"onnen aber ihre absolute Richtung nicht messen!
                    \item Es reicht nicht im Spiegel \textit{etwas anderes} zu sehen als im Labor, sofern es konsistent ist!
                \end{itemize}
            \end{itemize}}
		\end{column}
		\begin{column}{.4\textwidth}
			\centering
			\begin{overpic}[width=\textwidth,trim=150 0 150 0,clip]{img/rotated_coil.png}
            \end{overpic}
            \scalebox{.4}{(\textbf{Foto:} Z\'atonyi S\'andor)}
		\end{column}
    \end{columns}
    \only<3>{\tiny $\color{vertexDarkRed}{}^\star$zumindest nicht mit el.-magn. Effekten (Magnetischen Monopole ausgeschlossen)}
\end{frame}

\section{Wu-Experiment}

\begin{frame}{Das Wu-Experiment}
    \begin{block}{Physik im Jahre 1956}
        \begin{itemize}
            \item Parit\"at ist erhalten in 3 der 4 bekannten Wechselwirkungen:
            \begin{itemize}
                \item \textit{Gravitation}
                \item \textit{Elektromagnetische Wechselwirkung}
                \item \textit{Starke Wechselwirkung}
            \end{itemize}
            \item ... wie sieht es mit der \textbf{\textit{Schwachen Wechselwirkung}} aus?
            \begin{itemize}
                \item Pr\"apariere Kernzerfall: Cobalt-60 $\to$ Nickel-60 ($\beta$-Zerfall)
                \item Richte alle Kerne im \textcolor{vertexDarkRed}{magnetischen} Feld gleich aus$\color{vertexDarkRed}{}^\star$
                \item Untersuche die \textcolor{vertexDarkRed}{r\"aumliche Verteilung} der $\beta$-Strahlung
            \end{itemize}
        \end{itemize}
    \end{block}
    \tiny $\color{vertexDarkRed}{}^\star$als Auszeichnungsrichtung dient hier der sog. \textit{Kern-Spin}. Magn. Feld und \textit{Kern-Spin} sind nach Ausrichtung parallel.
\end{frame}

\begin{frame}{Einschub: Beta-Zerfall}
    \begin{itemize}
        \item \textbf{Atome} bestehen aus einem \textbf{Atomkern} und Elektronen
        \item \textbf{Atomkerne} sind aufgebaut aus \textbf{Protonen} und \textbf{Neutronen}
        \item Beim radioaktiven Zerfall wandelt sich ein Neutron in ein Proton \& Beta-Strahlung um (oder umgekehrt) 
        \begin{align*}
                n &\to p + \beta^- \quad \text{(Neutron-Zerfall)}\\
                p &\to n + \beta^+ \quad \text{(Proton-Zerfall)}
        \end{align*}
        \item \textbf{Beta-Strahlung} besteht aus zwei Teilchen: einem \textbf{Elektron} und einem \textbf{Anti-Elektron-Neutrino}
        \begin{align*}
            \beta^- &\;\hat{=}\; e^- + \overline{\nu}_e \quad \text{(Elektron + Anti-Elektron-Neutrino)}\\
            \beta^+ &\;\hat{=}\; e^+ + \nu_e \quad \text{(Positron + Elektron-Neutrino)}
        \end{align*}
    \end{itemize}
\end{frame}

\begin{frame}{Einschub: Beta-Zerfall}
    \textbf{Cobalt-60 Zerfall (nach Nickel-60)}
    \begin{itemize}
        \item Cobalt-60: 27 Protonen, 33 Neutronen
        \item Nickel-60: 28 Protonen, 32 Neutronen
        \item Also offenbar: Neutron $\to$ Proton ($\beta^-$ Strahlung, $\beta^- \;\hat{=}\; e^- + \overline{\nu}_e$)
        \item Das neue Proton nimmt den Platz vom zerfallenem Neutron ein. Das Elektron und das Anti-Elektron-Neutrino fliegen als $\beta^-$ Strahlung in \textbf{entgegengesetzte} Richtungen (Impulserhaltung).
    \end{itemize}
\end{frame}

\begin{frame}{Das Wu Experiment}
    \begin{columns}[T]
        \begin{column}{0.33\textwidth}
            \begin{overpic}[width=\textwidth]{img/wu.png}
                \put(20,50){\scalebox{5}{$\color{white}\uparrow$}}
                \put(30,20){\scalebox{2}{$\color{white}e^-$}}
                \put(40,90){\scalebox{2}{$\color{white}\overline{\nu}_{\!e}$}}
            \end{overpic}
            \center
            (Original)
        \end{column}
        \begin{column}{0.33\textwidth}
            \begin{overpic}[width=\textwidth]{img/wu_flipped.png}
                \put(37,50){\scalebox{4}{\color{white}?}}
                \put(35,20){\scalebox{2}{$\color{white}e^-$}}
                \put(20,90){\scalebox{2}{$\color{white}\overline{\nu}_{\!e}$}}
            \end{overpic}
            \center
            (Spiegelbild)
        \end{column}
        \begin{column}{0.33\textwidth}
            \begin{overpic}[width=\textwidth]{img/wu_flipped.png}
                \put(37,50){\scalebox{5}{$\color{white}\downarrow$}}
                \put(35,20){\scalebox{2}{$\color{white}\overline{\nu}_{\!e}$}}
                \put(20,90){\scalebox{2}{$\color{white}e^-$}}
            \end{overpic}
            \center
            (Nachbau)
        \end{column}
    \end{columns}
    \centering
    \scalebox{.4}{(\textbf{Foto:} Annual Report of the National Bureau of Standards for 1957, miscellaneous publication 227)}
\end{frame}

\begin{frame}{Das Wu-Experiment}
    \textbf{Das ist Parit\"atsverletzung!}
    \begin{itemize}
        \item Experimenteller Aufbau im Spiegel zeigt Elektronen Detektion \textbf{unten} (demnach m\"usste das magn. Feld nach \textbf{oben} zeigen)
        \item Nachbau des Spiegelbildes detektiert aber Elektronen \textbf{oben} (demnach m\"usste das magn. Feld nach \textbf{unten} zeigen)
        \item Warum?
        \begin{itemize}
            \item Im Vergleich zum Experiment mit dem Kompass sind das Elektron und das Anti-Elektron-Neutrino der Nord- und der S\"udpol des Kompasses. Diesmal aber \textbf{eindeutig bestimmt}!
        \end{itemize}
    \end{itemize}
\end{frame}

\begin{frame}{Das Wu-Experiment - Implikationen}
        Wir haben jetzt die M\"oglichkeit eine gespiegelte Welt \textbf{eindeutig zu definieren}:
        \begin{block}{Definition}
            Man lebt in einer gespiegelten Welt, wenn die Richtung der emittierten Elektronen im Cobalt-60 Kernzerf\"allen \textbf{parallel} (nicht anti-parallel) zu der Spin-Ausrichtung der Atomkerne ist.
        \end{block}
    \begin{itemize}
        \item Damit k\"onnen wir zwischen beiden Welten \textbf{unterscheiden} und ...
        \item ... wir k\"onnen \textbf{eindeutig bestimmen} in welcher Welt wir leben.
        \item Also ist unser Universum \textcolor{vertexDarkRed}{nicht parit\"atsinvariant}!
    \end{itemize}
\end{frame}

\begin{frame}{Das Wu-Experiment}
	\begin{columns}[T]
		\begin{column}{.6\textwidth}
            \textbf{Zusammenfassung}
            \begin{itemize}
                \item Schwache Wechselwirkung verletzt (als einzige) die Parit\"ats Invarianz
                \item Erstmals beobachtet mit dem \textit{Wu-Experiment}
                \item Parit\"ats Invarianz ist gebrochen, weil Beta-Strahlung parallel zur Magnetfeldrichtung emittiert wird
                \item Nobel Preis 1957
            \end{itemize}
		\end{column}
		\begin{column}{.4\textwidth}
			\centering
			\begin{overpic}[height=.8\textheight]{img/madamewu.png}
            \end{overpic}\\
            \centering
            \scalebox{.4}{(\textbf{Photo:} Acc. 90-105 - Science Service, Records, 1920s-1970s, Smithsonian Institution Archives)}
		\end{column}
	\end{columns}
\end{frame}

\section{Antimaterie}

\begin{frame}{Dirac-Gleichung}
    \begin{columns}[T]
        \begin{column}{.6\textwidth}
            \only<1>{%
            \textbf{Dirac-Gleichung}
            \begin{itemize}
                \item 1928 entdeckt Paul Dirac die \textit{Dirac-Gleichung}:
                \begin{center}
                    $(\mathrm{i} \slashed{\partial} - m) \psi = 0$ 
                \end{center}
                \item ... mit \textbf{2} (unabh\"angigen) L\"osungen
                \item Diese L\"osungen lassen sich \textit{physikalisch} als \textbf{Teilchen} und \textbf{Anti-Teilchen} interpretieren mit:
%                \item Die L\"osungen haben Eigenschaften welche sich physikalisch interpretieren lassen als:
%                \begin{itemize}
%                    \item Masse
%                    \item Elektrische Ladung
%                \end{itemize}
%                \item Diese 2 L\"osungen sind 2 \textit{Teilchen} mit
                \begin{itemize}
                    \item gleicher Masse
                    %\item gleicher, paarweise gespiegelter Ladung
                    \item paarweise gespiegelter Ladung
                \end{itemize}
                \item Beispiel (Elektron / Positron):
                \begin{itemize}
                    \item Elektron mit Ladung: $\mathbf{-1}$ (\textit{Teilchen})
                    \item Positron mit Ladung: $\mathbf{+1}$ (\textit{Anti-Teilchen})
                \end{itemize}
            \end{itemize}}
%                \begin{itemize}
%                    \item gleicher Masse
%                    \item gleicher Ladung
%                \end{itemize}
%            \end{itemize}}
%            \only<2,3>{%
%            \textbf{L\"osungen der Dirac-Gleichung}
%            \begin{itemize}
%                \item<2-> Beide L\"osungen haben die gleiche Masse, \textcolor{vertexDarkRed}{aber}
%                \begin{itemize}
%                    \item<2-> eine L\"osung hat \textbf{positive} Energie,
%                    \item<2-> eine L\"osung hat \textcolor{vertexDarkRed}{\textbf{negative}} Energie
%                \end{itemize}
%                \item<3> ... negative Energie?
%                \item<3> Entdeckung von R.~Feynman und E.~St\"uckelberg:
%                \begin{itemize}
%                    \item<3> Teilchen welche sich \textbf{r\"uckw\"arts} in der Zeit bewegen h\"atten negative Energie
%                    \item<3> Energie wird positiv, wenn wir stattdessen die Bewegungsrichtung und die Ladung spiegeln
%                    \item<3> diese Teilchen sind die \textbf{Anti-Teilchen}
%                \end{itemize}
%            \end{itemize}}
%            \only<4>{%
%            \textbf{L\"osungen der Dirac-Gleichung}
%            \begin{itemize}
%                \item Diese 2 L\"osungen sind 2 \textit{Teilchen} mit
%                \begin{itemize}
%                    \item gleicher Masse
%                    \item \sout{gleicher} paarweise gespiegelter Ladung
%                \end{itemize}
%                \item Beispiel (Elektron / Positron):
%                \begin{itemize}
%                    \item Elektron mit Ladung: $\mathbf{+1}$ (\textit{Teilchen})
%                    \item Positron mit Ladung: $\mathbf{-1}$ (\textit{Anti-Teilchen})
%                \end{itemize}
%            \end{itemize}}
%            \only<5>{%
            \only<2>{%
            \textbf{Zus\"atzliche Eigenschaft der L\"osungen}
            \begin{itemize}
                \item Ein \textit{Elektron} ist das Gleiche wie ein \textit{Positron}, wenn es sich 
                \begin{itemize}
                    \item in \textbf{gespiegelter} Richtung
                    \item mit \textbf{gespiegelter} Ladung bewegt
                \end{itemize}
                \item Mathematisch bewerkstelligt man die Umwandlung
                \begin{center}
                    Teilchen $\leftrightarrow$ Anti-Teilchen
                \end{center}
                durch Anwendung des sog. $C\!P$-Operators (Engl.: \textit{\underline{c}harge-\underline{p}arity})
            \end{itemize}}
        \end{column}
        \begin{column}{.4\textwidth}
            \centering
            \begin{overpic}[height=.7\textheight]{img/dirac.png}
            \end{overpic}
            \scalebox{.4}{(\textbf{Foto:} \texttt{http://nobelprize.org})}
        \end{column}
    \end{columns}    
\end{frame}

\begin{frame}{Antimaterie}
    \begin{columns}[T]
        \begin{column}{.6\textwidth}
            \textbf{Materie und Antimaterie}
            \begin{itemize}
                \item Jedes elementare Materieteilchen gehorcht seiner eigenen Dirac-Gleichung
                \begin{itemize}
                    \item Zu jedem Materieteilchen geh\"ort ein Materieantiteilchen (mit gleicher Masse und gespiegelter Ladung)
                \end{itemize}
                \item Ein Teilchen wird zum Anti-Teilchen durch Raum- und Ladungsspiegelung
            \end{itemize}
        \end{column}
        \begin{column}{.4\textwidth}
            \centering
            \begin{overpic}[height=.3\textheight,trim=400 400 400 100,clip]{img/donkey.png}
            \end{overpic}\\
            (Esel)\\
            \vspace*{.5cm}
            \reflectbox{%
            \begin{overpic}[height=.3\textheight,trim=400 400 400 100,clip]{img/anti_donkey.png}
            \end{overpic}}\\
            ($C\!P$-Esel)
        \end{column}
    \end{columns}    
\end{frame}

\begin{frame}{Einschub: Zusammensetzung unseres Universums}
    \begin{columns}[T]
        \begin{column}{.6\textwidth}
            \textbf{Zusammensetzung unseres Universums}
            \begin{itemize}
                \item 5\,\% Materie 
                \item 25\,\% Dunkle Materie (grav. Hinweise)
                \item 70\,\% Dunkle Energie (Ausdehnung v. Universum)
            \end{itemize}
            \hfill
            ... Antimaterie!?
        \end{column}
        \begin{column}{.4\textwidth}
            \centering
            \begin{overpic}[height=.7\textheight]{img/iceberg.png}
            \end{overpic} \\
            \scalebox{.4}{(\textbf{Bild:} Uwe Kils und Wiska Bodo)}
        \end{column}
    \end{columns}    
\end{frame}

\begin{frame}{Einschub: Zusammensetzung unseres Universums}
    \begin{columns}[T]
        \begin{column}{.6\textwidth}
            \textbf{Antimaterie}
            \begin{itemize}
                \item Antimaterie ist weder Dunkle Materie noch Dunkle Energie. Sie bildet eine eigene Kategorie und ist \textbf{sehr selten} in unserem Universum (sp\"ater dazu mehr).
                \item Produktion:
                \begin{itemize}
                    \item H\"ohenstrahlung
                    \item Teilchenbeschleuniger
                \end{itemize}
            \end{itemize}
        \end{column}
        \begin{column}{.4\textwidth}
           \centering
            \begin{overpic}[width=\textwidth,trim=300 0 400 0,clip]{img/antimatter_factory.png}
            \end{overpic}
        \end{column}
    \end{columns}    
\end{frame}

\begin{frame}{Das Wu-Experiment mit Antimaterie}
    \only<1>{%
    \begin{center}
        \scalebox{2}{\textbf{Ist $\pmb{C\!P}$ eine \textit{perfekte} Symmetrie?}}
    \end{center}}
    \only<2>{%
    \begin{columns}[T]
        \begin{column}{.6\textwidth}
            \textbf{Das Experiment im $\pmb{C\!P}$-Spiegel}
            \begin{itemize}
                \item Magnetische Spule
                \begin{itemize}
                    \item Windungsrichtung \textbf{und} Ladung werden gespiegelt
                    \item Magnetfeld \"andert seine Richtung nicht
                \end{itemize}
                \item Ausrichtung der Atome im magn. Feld:
                \begin{itemize}
                    \item h\"angt von elektrischer Ladung (inkl. Vorzeichen) ab
                    \item Magnetfeld und Atome sind jetzt also \textbf{antiparallel}
                \end{itemize}
            \end{itemize}
        \end{column}
        \begin{column}{.4\textwidth}
           \centering
            \begin{overpic}[height=.7\textheight]{img/madamewu.png}
            \end{overpic}\\
            \centering
            \scalebox{.4}{(\textbf{Photo:} Acc. 90-105 - Science Service, Records, 1920s-1970s, Smithsonian Institution Archives)}
        \end{column}
    \end{columns}}
\end{frame}

\begin{frame}{Das Wu-Experiment mit Antimaterie}
    \begin{columns}[T]
        \begin{column}{0.33\textwidth}
            \begin{overpic}[width=\textwidth]{img/wu.png}
                \put(20,50){\scalebox{5}{$\color{white}\uparrow$}}
                \put(30,20){\scalebox{2}{$\color{white}e^-$}}
                \put(40,90){\scalebox{2}{$\color{white}\overline{\nu}_{\!e}$}}
            \end{overpic}
            \center
            (Original)
        \end{column}
        \begin{column}{0.33\textwidth}
            \begin{overpic}[width=\textwidth]{img/anti_wu_flipped.png}
                \put(37,50){\scalebox{4}{\color{black}?}}
                \put(35,20){\scalebox{2}{$\color{black}e^+$}}
                \put(20,90){\scalebox{2}{$\color{black}\nu_{\!e}$}}
            \end{overpic}
            \center
            ($C\!P$-Spiegelbild)
        \end{column}
        \begin{column}{0.33\textwidth}
            \begin{overpic}[width=\textwidth]{img/anti_wu_flipped.png}
                \put(37,50){\scalebox{5}{$\color{black}\uparrow$}}
                \put(35,20){\scalebox{2}{$\color{black}e^+$}}
                \put(20,90){\scalebox{2}{$\color{black}\nu_{\!e}$}}
            \end{overpic}
            \center
            (Nachbau)
        \end{column}
    \end{columns}
    \centering
    \scalebox{.4}{(\textbf{Foto:} Annual Report of the National Bureau of Standards for 1957, miscellaneous publication 227)}
\end{frame}

\begin{frame}{Das Wu-Experiment mit Antimaterie}
    \begin{columns}[T]
        \begin{column}{.6\textwidth}
            \textbf{Auswertung}
            \begin{itemize}
                \item Problem im (Parit\"ats-)Spiegel
                \begin{itemize}
                    \item die Atomausrichtung kehrt sich um (umgekehrte Magnetfeldrichtung)
                    \item $e^-$-Emission \textbf{entgegen} der Atomausrichtung, sowohl im Original, als auch im Nachbau
                \end{itemize}
                \item<2> ... im $C\!P$-Spiegel
                \begin{itemize}
                    \item<2> die (Anti-)Atomausrichtung kehrt sich um (gleiche Magnetfeldrichtung)
                    \item<2> $e^+$-Emission \textbf{entlang} der \textbf{Anti-}Atomausrichtung
                \end{itemize}
            \end{itemize}
        \end{column}
        \begin{column}{.4\textwidth}
            \centering
            \only<1>{%
            \begin{overpic}[height=.7\textheight]{img/wu.png}
                \put(20,50){\scalebox{5}{$\color{white}\uparrow$}}
                \put(30,20){\scalebox{2}{$\color{white}e^-$}}
                \put(40,90){\scalebox{2}{$\color{white}\overline{\nu}_{\!e}$}}
            \end{overpic}\\
            (Original)}%
            \only<2>{%
            \begin{overpic}[height=.7\textheight]{img/anti_wu_flipped.png}
                \put(37,50){\scalebox{5}{$\color{black}\uparrow$}}
                \put(35,20){\scalebox{2}{$\color{black}e^+$}}
                \put(20,90){\scalebox{2}{$\color{black}\nu_{\!e}$}}
            \end{overpic}\\
            (Nachbau)}
        \end{column}
    \end{columns}
\end{frame}

\begin{frame}{Das Wu-Experiment mit Antimaterie}
    \begin{columns}[T]
        \begin{column}{.6\textwidth}
            \textbf{Gedankenspiel}: Kommunikation mit einem Alien
            \begin{itemize}
                \item Wir wollen herausfinden ob sein Universum \textit{\sout{gespiegelt} aus Antimaterie aufgebaut} ist ...
                \item ... \textbf{ohne} vorbei fliegen zu m\"ussen.
                \item K\"onnen wir ihm ein Experiment vorschlagen mit dem wir unsere beiden Definitionen von \textbf{\sout{links} Materie} und \textbf{\sout{rechts} Antimaterie} vergleichen k\"onnen?
                \begin{itemize}
                    \item Offenbar nicht mit dem Wu-Experiment (mit Antimaterie)!
					\item Ist unsere Definition von Materie und Antimaterie als \textbf{willk\"urlich}?
                \end{itemize}
            \end{itemize}
        \end{column}
        \begin{column}{.4\textwidth}
            \centering
            \begin{overpic}[height=.7\textheight]{img/et.png}
            \end{overpic}\\
            \scalebox{.4}{(\textbf{Poster:} John Alvin. (c) 1982 Universal Studios / \textit{low-resolution})}
        \end{column}
    \end{columns}
\end{frame}

\section{CP-Verletzung}

\begin{frame}{$C\!P$-Verletzung}
    \begin{columns}[T]
        \begin{column}{.6\textwidth}
            \begin{itemize}
                \item Ein Blick in unser Universum
                \begin{itemize}
                    \item (fast) ausschlie\ss{}lich Materie und keine Antimaterie
                \end{itemize}
                \item Standard Modell der Kosmologie
                \begin{itemize}
                    \item Beim Urknall wurden exakt gleiche Mengen von Materie und Antimaterie produziert
                \end{itemize}
            \end{itemize}
            \centering
            \textbf{Wo ist die Antimaterie?}
        \end{column}
        \begin{column}{.4\textwidth}
            \centering
            \begin{overpic}[height=.7\textheight]{img/ams02.png}
            \end{overpic}\\
            \centering
            \scalebox{.4}{(\textbf{Logo:} AMS-02, NASA/JSC)}
        \end{column}
    \end{columns}
\end{frame}

\begin{frame}{$C\!P$-Verletzung}
    \begin{columns}[T]
        \begin{column}{.6\textwidth}
            \textbf{Wo ist die Antimaterie?}
            \begin{itemize}
                \item<1-> da wir sie im Universum nicht finden, muss sie bereits zerfallen sein!
                \item<1-> \textbf{$\pmb{C\!P}$-Verletzung!}
                \item<1-> ... nur wo?
                \item<2-> Standard-Modell der Teilchenphysik (SM)
                \begin{itemize}
                    \item<2-> beste best\"atigste Theorie f\"ur 3/4 fundamentalen Wechselwirkungen (Gravitation nicht enthalten)
                    \item<2-> enth\"alt einen Sektor mit $C\!P$-Verletzung
                \end{itemize}
            \end{itemize}
        \end{column}
        \begin{column}{.4\textwidth}
            \centering
            \only<2>{%
            \begin{overpic}[height=.7\textheight]{img/cup.png}
            \end{overpic}}
        \end{column}
    \end{columns}
\end{frame}

\begin{frame}{$C\!P$-Verletzung}
    \begin{columns}[T]
        \begin{column}{.6\textwidth}
            \textbf{$\pmb{C\!P}$-Verletzung - das Problem}
            \begin{itemize}
                \item $C\!P$-Verletzung ist im SM um \textbf{viele Gr\"o\ss{}enordnungen zu klein} um die Asymmetrie im Universum erkl\"aren zu k\"onnen
                \item Dennoch: bis heute \textbf{einzige} Erkl\"arung f\"ur $C\!P$-Verletzung
                \item ... ist die Theorie richtig?
                \begin{itemize}
                    \item keine (signifikanten) Abweichungen bis zu erreichbaren Energien bekannt
                    \item vermutlich falsch f\"ur sehr gro\ss{}e Energien (keiner wei\ss{} genau was in diesem Kontext \enquote{gro\ss{}} bedeutet)
                \end{itemize}
            \end{itemize}
        \end{column}
        \begin{column}{.4\textwidth}
            \centering
            \begin{overpic}[height=.7\textheight]{img/cup.png}
            \end{overpic}
        \end{column}
    \end{columns}
\end{frame}

\begin{frame}{$C\!P$-Verletzung}
    \begin{columns}[T]
        \begin{column}{.5\textwidth}
            \textbf{$\pmb{C\!P}$-Verletzung - ein L\"osungsansatz}
            \begin{itemize}
                \item Pr\"azisionsmessungen
                \begin{itemize}
                    \item Vorhersagen vom SM werden verbessert
                    \item finden wir Abweichungen / Hinweise auf \textit{neue Physik?}
                \end{itemize}
                \item Zur Zeit beste$\color{vertexDarkRed}{}^\star$ Experiment: \textbf{LHCb} (CERN)
                \begin{itemize}
                    \item Detektor am LHC-Speicherring
                \end{itemize}
            \end{itemize}
        \end{column}
        \begin{column}{.5\textwidth}
            \centering
            \begin{overpic}[width=\textwidth]{img/lhcb_collaboration.png}
            \end{overpic}
        \end{column}
    \end{columns}
    \tiny $\color{vertexDarkRed}{}^\star$pers\"onliche Meinung des Referenten
\end{frame}

%\begin{frame}{Das LHCb-Experiment}
%    \begin{columns}[T]
%        \begin{column}{.5\textwidth}
%            \textbf{Der Large Hadron Collider (LHC)}
%            \begin{itemize}
%                \item Wikipedia: \enquote{[...] the most complex experimental facility ever built and the largest single machine in the world.}
%                \item Synchrotron (in einem 27\,km langem unterirdischem Ringtunnel): \textbf{beschleunigt} u.a. Protonen auf fast Lichtgeschwindigkeit und \textbf{kollidiert} diese an bestimmten Punkten (zum Beispiel beim LHCb Detektor)
%            \end{itemize}
%        \end{column}
%        \begin{column}{.5\textwidth}
%            \centering
%            \begin{overpic}[width=\textwidth]{img/cern_map.png}
%            \end{overpic}
%        \end{column}
%    \end{columns}
%\end{frame}

\begin{frame}{Das LHCb-Experiment}
    \enquote{LHCb is an experiment set up to explore what happened after the Big Bang that allowed matter to survive and build the Universe we inhabit today}\\
    {\footnotesize (\texttt{http://lhcb-public.web.cern.ch})}
    \begin{columns}[T]
        \begin{column}{.7\textwidth}
            \begin{itemize}
                \item u.a.: Spezialisiert auf die Messung von $C\!P$-Verletzung in $b$-Hadron Zerf\"allen
                \item >\num{442} Papers und >\num{23281} Zitationen$\color{vertexDarkRed}{}^\star$
                \item ... bis jetzt (leider?) noch keine signifikante Abweichung vom SM gefunden
            \end{itemize}
        \end{column}
        \begin{column}{.3\textwidth}
            \centering
            \begin{overpic}[height=.7\textheight]{img/lhcb_pit.png}
            \end{overpic}
        \end{column}
    \end{columns}
    \tiny $\color{vertexDarkRed}{}^\star$Stand 08.06.2018
\end{frame}

\begin{frame}{Das LHCb-Experiment}
    \begin{center}
        \begin{overpic}[width=.7\textwidth]{img/lhcb_detector.png}
        \end{overpic}
    \end{center}

    \begin{itemize}
        \item Im deutschen Wikipedia Artikel zum \textit{Large Hadron Collider}$\color{vertexDarkRed}{}^\star$ werden zwei Ergebnisse des LHCb-Experimentes noch vor der Entdeckung des Higgs-Bosons aufgelistet (Nobelpreis 2013)
        \begin{itemize}
            \item darunter: erstmalige Messung von \underline{$C\!P$-Verletzung} bei $B_s^0$-Meson Zerf\"allen
        \end{itemize}
    \end{itemize}
\end{frame}

\begin{frame}{Das LHCb-Experiment}
    \textbf{Beispiel:} Messung von $C\!P$-Verletzung in Zerf\"allen schwerer $b$-Mesonen\\
    {\footnotesize(Phys. Lett. \textbf{B777} (2018) 16-30)}
    \begin{columns}[T]
        \begin{column}{.7\textwidth}
            \begin{overpic}[width=\textwidth]{img/cpv_Bu2DK.png}
                \linethickness{0.3mm}
                \put(31,42){\line(1,0){45.5}}
            \end{overpic}
        \end{column}
        \begin{column}{.3\textwidth}
            \begin{overpic}[width=\textwidth]{img/cpv_Bu2DK_key.png}
            \end{overpic}
        \end{column}
    \end{columns}
    \begin{itemize}
        \item Aug. 2017: CP-Verletzung in den Zerf\"allen $B^\pm \to D^0 h^\pm$. Der Unterschied kann mit blo\ss{}em Auge gesehen werden ...
    \end{itemize}
\end{frame}

\begin{frame}{Das LHCb-Experiment}
    \begin{columns}[T]
        \begin{column}{.6\textwidth}
            \textbf{Beispiel:} System for Measuring Overlap with Gas (SMOG)
            \begin{itemize}
                \item AMS-02: misst unerwartet viele $\overline{p}$ bei hohen Energien ($\gtrapprox$\,50\,GeV)
                \item LHCb: SMOG sensitiv f\"ur Energien 10-100\,GeV (in $p$He-Kollisionen) und kann damit die Ergebnisse von AMS-02 best\"atigen / verbessern
            \end{itemize}
        \end{column}
        \begin{column}{.4\textwidth}
            \begin{overpic}[width=\textwidth]{img/ams02_results.png}
            \end{overpic}\\
            \centering
            \scalebox{.4}{(\textbf{Plot:} \textbf{JCAP09} (2015) 023)}
        \end{column}
    \end{columns}
\end{frame}

\begin{frame}{Zusammenfassung}
    \textbf{Zusammenfassung}
    \begin{itemize}
        \item Materie und Antimaterie verhalten sich minimal unterschiedlich. Dieses Ph\"anomen hei\ss{}t $C\!P$-Verletzung.
        \item Dieser kleiner Unterschied ist extrem aufwendig zu vermessen ...
        \item ... k\"onnte aber helfen zu verstehen warum wir \"uberhaupt noch existieren.
        \item Am LHCb-Experiment (CERN) vermessen wir (unteranderem) diese Materie-Antimaterie-Asymmetrie mit noch nie da gewesener Pr\"azision.
    \end{itemize}
    \begin{center}
        \textbf{Vielen Dank f\"ur Ihre Aufmerksamkeit!}
    \end{center}
    \begin{textblock*}{5cm}(11cm,7cm)
        \centering
        \begin{overpic}[height=2cm]{img/dracula.png}
        \end{overpic}\\
        \scalebox{.4}{(\textbf{Bild:} \texttt{https://www.lego.com} / \textit{low-resolution})}
    \end{textblock*}
\end{frame}


\end{document}

