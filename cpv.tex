\section{CP-Verletzung}

\begin{frame}{$C\!P$-Verletzung}
    \begin{columns}[T]
        \begin{column}{.6\textwidth}
            \begin{itemize}
                \item Ein Blick in unser Universum
                \begin{itemize}
                    \item (fast) ausschlie\ss{}lich Materie und keine Antimaterie
                \end{itemize}
                \item Standard Modell der Kosmologie
                \begin{itemize}
                    \item Beim Urknall wurden exakt gleiche Mengen von Materie und Antimaterie produziert
                \end{itemize}
            \end{itemize}
            \centering
            \textbf{Wo ist die Antimaterie?}
        \end{column}
        \begin{column}{.4\textwidth}
            \centering
            \begin{overpic}[height=.7\textheight]{img/ams02.png}
            \end{overpic}\\
            \centering
            \scalebox{.4}{(\textbf{Logo:} AMS-02, NASA/JSC)}
        \end{column}
    \end{columns}
\end{frame}

\begin{frame}{$C\!P$-Verletzung}
    \begin{columns}[T]
        \begin{column}{.6\textwidth}
            \textbf{Wo ist die Antimaterie?}
            \begin{itemize}
                \item<1-> da wir sie im Universum nicht finden, muss sie bereits zerfallen sein!
                \item<1-> \textbf{$\pmb{C\!P}$-Verletzung!}
                \item<1-> ... nur wo?
                \item<2-> Standard-Modell der Teilchenphysik (SM)
                \begin{itemize}
                    \item<2-> beste best\"atigste Theorie f\"ur 3/4 fundamentalen Wechselwirkungen (Gravitation nicht enthalten)
                    \item<2-> enth\"alt einen Sektor mit $C\!P$-Verletzung
                \end{itemize}
            \end{itemize}
        \end{column}
        \begin{column}{.4\textwidth}
            \centering
            \only<2>{%
            \begin{overpic}[height=.7\textheight]{img/cup.png}
            \end{overpic}}
        \end{column}
    \end{columns}
\end{frame}

\begin{frame}{$C\!P$-Verletzung}
    \begin{columns}[T]
        \begin{column}{.6\textwidth}
            \textbf{$\pmb{C\!P}$-Verletzung - das Problem}
            \begin{itemize}
                \item $C\!P$-Verletzung ist im SM um \textbf{viele Gr\"o\ss{}enordnungen zu klein} um die Asymmetrie im Universum erkl\"aren zu k\"onnen
                \item Dennoch: bis heute \textbf{einzige} Erkl\"arung f\"ur $C\!P$-Verletzung
                \item ... ist die Theorie richtig?
                \begin{itemize}
                    \item keine (signifikanten) Abweichungen bis zu erreichbaren Energien bekannt
                    \item vermutlich falsch f\"ur sehr gro\ss{}e Energien (keiner wei\ss{} genau was in diesem Kontext \enquote{gro\ss{}} bedeutet)
                \end{itemize}
            \end{itemize}
        \end{column}
        \begin{column}{.4\textwidth}
            \centering
            \begin{overpic}[height=.7\textheight]{img/cup.png}
            \end{overpic}
        \end{column}
    \end{columns}
\end{frame}

\begin{frame}{$C\!P$-Verletzung}
    \begin{columns}[T]
        \begin{column}{.5\textwidth}
            \textbf{$\pmb{C\!P}$-Verletzung - ein L\"osungsansatz}
            \begin{itemize}
                \item Pr\"azisionsmessungen
                \begin{itemize}
                    \item Vorhersagen vom SM werden verbessert
                    \item finden wir Abweichungen / Hinweise auf \textit{neue Physik?}
                \end{itemize}
                \item Zur Zeit beste$\color{vertexDarkRed}{}^\star$ Experiment: \textbf{LHCb} (CERN)
                \begin{itemize}
                    \item Detektor am LHC-Speicherring
                \end{itemize}
            \end{itemize}
        \end{column}
        \begin{column}{.5\textwidth}
            \centering
            \begin{overpic}[width=\textwidth]{img/lhcb_collaboration.png}
            \end{overpic}
        \end{column}
    \end{columns}
    \tiny $\color{vertexDarkRed}{}^\star$pers\"onliche Meinung des Referenten
\end{frame}

%\begin{frame}{Das LHCb-Experiment}
%    \begin{columns}[T]
%        \begin{column}{.5\textwidth}
%            \textbf{Der Large Hadron Collider (LHC)}
%            \begin{itemize}
%                \item Wikipedia: \enquote{[...] the most complex experimental facility ever built and the largest single machine in the world.}
%                \item Synchrotron (in einem 27\,km langem unterirdischem Ringtunnel): \textbf{beschleunigt} u.a. Protonen auf fast Lichtgeschwindigkeit und \textbf{kollidiert} diese an bestimmten Punkten (zum Beispiel beim LHCb Detektor)
%            \end{itemize}
%        \end{column}
%        \begin{column}{.5\textwidth}
%            \centering
%            \begin{overpic}[width=\textwidth]{img/cern_map.png}
%            \end{overpic}
%        \end{column}
%    \end{columns}
%\end{frame}

\begin{frame}{Das LHCb-Experiment}
    \enquote{LHCb is an experiment set up to explore what happened after the Big Bang that allowed matter to survive and build the Universe we inhabit today}\\
    {\footnotesize (\texttt{http://lhcb-public.web.cern.ch})}
    \begin{columns}[T]
        \begin{column}{.7\textwidth}
            \begin{itemize}
                \item u.a.: Spezialisiert auf die Messung von $C\!P$-Verletzung in $b$-Hadron Zerf\"allen
                \item >\num{442} Papers und >\num{23281} Zitationen$\color{vertexDarkRed}{}^\star$
                \item ... bis jetzt (leider?) noch keine signifikante Abweichung vom SM gefunden
            \end{itemize}
        \end{column}
        \begin{column}{.3\textwidth}
            \centering
            \begin{overpic}[height=.7\textheight]{img/lhcb_pit.png}
            \end{overpic}
        \end{column}
    \end{columns}
    \tiny $\color{vertexDarkRed}{}^\star$Stand 08.06.2018
\end{frame}

\begin{frame}{Das LHCb-Experiment}
    \begin{center}
        \begin{overpic}[width=.7\textwidth]{img/lhcb_detector.png}
        \end{overpic}
    \end{center}

    \begin{itemize}
        \item Im deutschen Wikipedia Artikel zum \textit{Large Hadron Collider}$\color{vertexDarkRed}{}^\star$ werden zwei Ergebnisse des LHCb-Experimentes noch vor der Entdeckung des Higgs-Bosons aufgelistet (Nobelpreis 2013)
        \begin{itemize}
            \item darunter: erstmalige Messung von \underline{$C\!P$-Verletzung} bei $B_s^0$-Meson Zerf\"allen
        \end{itemize}
    \end{itemize}
\end{frame}

\begin{frame}{Das LHCb-Experiment}
    \textbf{Beispiel:} Messung von $C\!P$-Verletzung in Zerf\"allen schwerer $b$-Mesonen\\
    {\footnotesize(Phys. Lett. \textbf{B777} (2018) 16-30)}
    \begin{columns}[T]
        \begin{column}{.7\textwidth}
            \begin{overpic}[width=\textwidth]{img/cpv_Bu2DK.png}
                \linethickness{0.3mm}
                \put(31,42){\line(1,0){45.5}}
            \end{overpic}
        \end{column}
        \begin{column}{.3\textwidth}
            \begin{overpic}[width=\textwidth]{img/cpv_Bu2DK_key.png}
            \end{overpic}
        \end{column}
    \end{columns}
    \begin{itemize}
        \item Aug. 2017: CP-Verletzung in den Zerf\"allen $B^\pm \to D^0 h^\pm$. Der Unterschied kann mit blo\ss{}em Auge gesehen werden ...
    \end{itemize}
\end{frame}

\begin{frame}{Das LHCb-Experiment}
    \begin{columns}[T]
        \begin{column}{.6\textwidth}
            \textbf{Beispiel:} System for Measuring Overlap with Gas (SMOG)
            \begin{itemize}
                \item AMS-02: misst unerwartet viele $\overline{p}$ bei hohen Energien ($\gtrapprox$\,50\,GeV)
                \item LHCb: SMOG sensitiv f\"ur Energien 10-100\,GeV (in $p$He-Kollisionen) und kann damit die Ergebnisse von AMS-02 best\"atigen / verbessern
            \end{itemize}
        \end{column}
        \begin{column}{.4\textwidth}
            \begin{overpic}[width=\textwidth]{img/ams02_results.png}
            \end{overpic}\\
            \centering
            \scalebox{.4}{(\textbf{Plot:} \textbf{JCAP09} (2015) 023)}
        \end{column}
    \end{columns}
\end{frame}
