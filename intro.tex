\begin{frame}{Was ist der LHC / Was ist CERN?}
    \begin{columns}[T]
        \begin{column}{.5\textwidth}
            \textbf{Der Large Hadron Collider (LHC)}
            \begin{itemize}
                \item Wikipedia: \enquote{[...] the most complex experimental facility ever built and the largest single machine in the world.}
                \item Synchrotron (in einem 27\,km langem unterirdischem Ringtunnel): \textbf{beschleunigt} u.a. Protonen auf fast Lichtgeschwindigkeit und \textbf{kollidiert} diese an bestimmten Punkten (zum Beispiel beim LHCb Detektor)
            \end{itemize}
        \end{column}
        \begin{column}{.5\textwidth}
            \centering
            \begin{overpic}[width=\textwidth]{img/cern_map.png}
            \end{overpic}
        \end{column}
    \end{columns}
\end{frame}

\begin{frame}{Was wir nicht verstehen}
    \begin{columns}[T]
        \begin{column}{.6\textwidth}
            \begin{itemize}
                \item Ein Blick in unser Universum
                \begin{itemize}
                    \item (fast) ausschlie\ss{}lich Materie und keine Antimaterie
                \end{itemize}
                \item Standard Modell der Kosmologie
                \begin{itemize}
                    \item Beim Urknall wurden exakt gleiche Mengen von Materie und Antimaterie produziert
                \end{itemize}
            \end{itemize}
            \centering
            \textbf{Wo ist die Antimaterie?}
        \end{column}
        \begin{column}{.4\textwidth}
            \centering
            \begin{overpic}[height=.7\textheight]{img/ams02.png}
            \end{overpic}\\
            \centering
            \scalebox{.4}{(\textbf{Logo:} AMS-02, NASA/JSC)}
        \end{column}
    \end{columns}
\end{frame}
